\documentclass{article}
\usepackage[utf8]{inputenc}
\usepackage{ragged2e}
\usepackage{geometry}
\usepackage{pdfpages}
\usepackage{biblatex}
\geometry{portrait}

\addbibresource{main.bib}

\title{Scientific Calculator}
\author{Anand Kacha (40047673)}

\begin{document}

\tableofcontents
\listoftables

\clearpage

\section{User Stories}
\begin{flushleft}
\textbf{Note:} The reference point for the user story estimate is 13 points for 1 day.
\end{flushleft}

\subsection{Global Constraints}
\begin{itemize}
    \item The system takes the text based inputs from the command line console.
    \item The system throws error on wrongly formatted input.
\end{itemize}

\subsection{User Stories}
\begin{table}[h]
\centering
\begin{tabular}{|p{2.2cm}|p{12cm}|}
\hline
\textbf{Identifier} & {calc-us-001} \\
\hline
\textbf{Description} & {A user can calculate basic equations related to arithmetic operators.}\\
\hline
\textbf{Constraints} & 
\begin{enumerate}
    \item The user can use number pad to input the digits.
    \item The user can use symbols to input the special constants.
    \item The operations involve both numbers and symbols.
\end{enumerate}\\
\hline
\textbf{Acceptance Tests} & 
\begin{enumerate}
    \item The user is able to perform arithmetic operations between two numbers.
    \item The user is able to perform arithmetic operations between symbols.
    \item The user is able to perform arithmetic operations amongst number and symbols.
\end{enumerate}\\
\hline
\textbf{Priority} & {High}\\
\hline
\textbf{Estimate} & {5 points}\\
\hline
\end{tabular}
\caption{User Story 1}
\end{table}

\begin{table}[h]
\centering
\begin{tabular}{|p{2.2cm}|p{12cm}|}
\hline
\textbf{Identifier} & {calc-us-002} \\
\hline
\textbf{Description} & {A user can evaluate expressions with scientific operators.}\\
\hline
\textbf{Constraints} & 
\begin{enumerate}
    \item The user can combine two different kind of scientific operators.
\end{enumerate}\\
\hline
\textbf{Acceptance Tests} & 
\begin{enumerate}
    \item The user is able to perform logarithmic and trigonometric operations.
    \item The user is able to use arithmetic operators with scientific operators.
\end{enumerate}\\
\hline
\textbf{Priority} & {Medium}\\
\hline
\textbf{Estimate} & {8 points}\\
\hline
\end{tabular}
\caption{User Story 2}
\end{table}

\begin{table}[h]
\centering
\begin{tabular}{|p{2.2cm}|p{12cm}|}
\hline
\textbf{Identifier} & {calc-us-003} \\
\hline
\textbf{Description} & {A user can evaluate an expression in order to process with variable data for his research.}\\
\hline
\textbf{Constraints} & 
\begin{enumerate}
    \item The expression can be the combination of arithmetic operations and logarithmic operations.
    \item The expression may contain the Gelfond's constant.
\end{enumerate}\\
\hline
\textbf{Acceptance Tests} & 
\begin{enumerate}
    \item The user is able to calculate the roots of an equation.
\end{enumerate}\\
\hline
\textbf{Priority} & {Low}\\
\hline
\textbf{Estimate} & {13 points}\\
\hline
\end{tabular}
\caption{User Story 3}
\end{table}

\begin{table}[h]
\centering
\begin{tabular}{|p{2.2cm}|p{12cm}|}
\hline
\textbf{Identifier} & {calc-us-004} \\
\hline
\textbf{Description} & {A user can calculate the volume of n-ball sphere.}\\
\hline
\textbf{Constraints} & 
\begin{enumerate}
    \item The user will provide only the value of n.
    \item The sphere will always be unit size. 
\end{enumerate}\\
\hline
\textbf{Acceptance Tests} & 
\begin{enumerate}
    \item The user can validate the size of the balls to be unit.
    \item The user can calculate the volume.
\end{enumerate}\\
\hline
\textbf{Priority} & {Medium}\\
\hline
\textbf{Estimate} & {21 points}\\
\hline
\end{tabular}
\caption{User Story 4}
\end{table}

\begin{table}[h]
\centering
\begin{tabular}{|p{2.2cm}|p{12cm}|}
\hline
\textbf{Identifier} & {calc-us-005} \\
\hline
\textbf{Description} & {A user can derive almost integer numbers.}\\
\hline
\textbf{Constraints} & 
\begin{enumerate}
    \item The user can use only transcendental numbers in order to mix it with the Gelfond's constant.
\end{enumerate}\\
\hline
\textbf{Acceptance Tests} & 
\begin{enumerate}
    \item The user should be able to generate Ramanujan's constant, $e^{\pi} - \pi$
\end{enumerate}\\
\hline
\textbf{Priority} & {Medium}\\
\hline
\textbf{Estimate} & {13 points}\\
\hline
\end{tabular}
\caption{User Story 5}
\end{table}

\clearpage
\section{Backward Traceability Matrix}
\subsection{Matrix}
\begin{table}[h]
\centering
\begin{tabular}{|p{2cm}|p{2cm}|p{2cm}|p{2.2cm}|}
\hline
\textbf{} & \textbf{Use Cases} & \textbf{Persona} & \textbf{Applications} \\
\hline
{calc-us-1} & {calc-uc-1} & {} & {}\\
\hline
{calc-us-2} & {calc-uc-1} & {1} & {}\\
\hline
{calc-us-3} & {calc-uc-1} & {} & {}\\
\hline
{calc-us-4} & {calc-uc-3} & {} & {1}\\
\hline
{calc-us-5} & {calc-uc-2} & {} & {1}\\
\hline
\end{tabular}
\caption{Backward Traceability Matrix}
\end{table}

\subsection{Description}
\justifying
The first column represents the use case identifiers. The use case identifiers are addressed in the User Stories section.\newline \newline
In the use cases column, \textbf{calc-uc-1} corresponds to "Evaluate Basic Expressions". \textbf{calc-uc-2} corresponds to "Calculate Almost Integer Number". \textbf{calc-uc-3} corresponds to "Calculate volume of u-unit balls"

\end{document}
