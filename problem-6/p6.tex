\documentclass{article}
\usepackage[utf8]{inputenc}
\usepackage{ragged2e}
\usepackage{geometry}
\geometry{letter, portrait}

\title{User Stories}
\author{Anand Kacha : 40047673}
\date{July 26, 2019}

\begin{document}

\maketitle

\begin{flushleft}
\textbf{Note:} The reference point for the user story estimate is 13 points for 1 day.
\end{flushleft}

\begin{flushleft}
\textbf{[calc-us-001]}A user can calculate basic equations related to arithmetic operators.
\end{flushleft}
\textit{Constraints}
\begin{enumerate}
    \item The user can use number pad to input the digits.
    \item The user can use symbols to input the special constants.
    \item The operations involve both numbers and symbols.
\end{enumerate}
\textit{Acceptance Criteria}
\begin{enumerate}
    \item The user is able to perform arithmetic operations between two numbers.
    \item The user is able to perform arithmetic operations between symbols.
    \item The user is able to perform arithmetic operations amongs number and symbols.
\end{enumerate}
\textit{Priority}: high. \newline
\textit{Estimate}: 5 points.

\begin{flushleft}
\textbf{[calc-us-002]} use can evaluate expressions with scientific operators.
\end{flushleft}
\textit{Constraints}
\begin{enumerate}
    \item The user can combine two different kind of scientific operators.
\end{enumerate}
\newline
\textit{Acceptance Criteria}
\begin{enumerate}
    \item The user is able to perform logarithmic and trigonometric operations.
    \item The user is able to use arithmetic operators with scientific operators.
\end{enumerate}
\textit{Priority}: medium \newline
\textit{Estimate}: 8 points.

\begin{flushleft}
\textbf{[calc-us-003]} A user can evaluate an expression in order to process with variable data for his research.
\end{flushleft}
\textit{Constraints}
\begin{enumerate}
    \item The expression can be the combination of arithmetic operations and logarithmic operations.
    \item The expression may contain the Gelfond's constant.
\end{enumerate}
\textit{Acceptance Criteria}
\begin{enumerate}
    \item The user is able to calculate the roots of an equation.
\end{enumerate}
\textit{Priority}: low \newline
\textit{Estimate}: 13 points.

\begin{flushleft}
\textbf{[calc-us-004]} A user can calculate the volume of n-ball sphere.
\end{flushleft}
\textit{Constraints}
\begin{enumerate}
    \item The user will provide only the value of n.
    \item The sphere will always be unit size. 
\end{enumerate}
\textit{Acceptance Criteria}
\begin{enumerate}
    \item The user can validate the size of the balls to be unit.
    \item The user can calculate the volume.
\end{enumerate}
\textit{Priority}: medium \newline
\textit{Estimate}: 21 points.

\begin{flushleft}
\textbf{[calc-us-005]} A user can derive almost integer numbers.
\end{flushleft}
\textit{Constraints}
\begin{enumerate}
    \item The user can use only transcendental numbers in order to mix it with the Gelfond's constant.
\end{enumerate}
\textit{Acceptance Criteria}
\begin{enumerate}
    \item The user should be able to generate Ramanujan's constant, $e^{\pi} - \pi$
\end{enumerate}
\textit{Priority}: medium \newline
\textit{Estimate}: 13 points.

\end{document}
